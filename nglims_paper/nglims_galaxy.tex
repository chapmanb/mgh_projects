\NeedsTeXFormat{LaTeX2e}[1995/12/01]
\documentclass[10pt]{bmc_article}

\usepackage{cite} % Make references as [1-4], not [1,2,3,4]
\usepackage{url}  % Formatting web addresses
\usepackage{ifthen}  % Conditional
\usepackage{multicol}   %Columns
\usepackage[utf8]{inputenc} %unicode support
%\usepackage[applemac]{inputenc} %applemac support if unicode package fails
%\usepackage[latin1]{inputenc} %UNIX support if unicode package fails
\usepackage{hyperref} % clickable links, Brad
\urlstyle{rm}

\def\includegraphic{}
\def\includegraphics{}


\setlength{\topmargin}{0.0cm}
\setlength{\textheight}{21.5cm}
\setlength{\oddsidemargin}{0cm}
\setlength{\textwidth}{16.5cm}
\setlength{\columnsep}{0.6cm}

\newboolean{publ}

%Review style settings
\newenvironment{bmcformat}{\begin{raggedright}\baselineskip20pt\sloppy\setboolean{publ}{false}}{\end{raggedright}\baselineskip20pt\sloppy}

%Publication style settings
%\newenvironment{bmcformat}{\fussy\setboolean{publ}{true}}{\fussy}

\begin{document}
\begin{bmcformat}


\title{Next generation sequencing information management and analysis
  system for Galaxy}

\author{Brad A Chapman\correspondingauthor$^{1,3}$%
       \email{chapman@molbio.mgh.harvard.edu}%
      and
         Mark L Borowsky$^{1,2}$%
         \email{borowsky@molbio.mgh.harvard.edu}%
      }

\address{%
    \iid(1)Department of Molecular Biology, Simches Research Center,%
    Massachusetts General Hospital, Boston, MA 02114, USA
    \iid(2)Department of Genetics,%
    Harvard Medical School, Boston, MA, USA
    \iid(3)Harvard School of Public Health,%
    SPH2, Room 437A, 655 Huntington Ave  Boston, MA 02115, USA
}%

\maketitle

\begin{abstract}
  \paragraph*{Background:} Next generation sequencing technologies
  like Illumina, SOLiD and 454 provide small core facilities and
  individual laboratories with the ability to produce large amounts of sequence
  data. Along with this increased output comes the challenge of
  managing samples, tracking sequencing runs, and automating downstream
  analyses.
  \paragraph*{Results:} We approached these challenges by developing a
  sample submission and tracking interface on top of the web-based
  Galaxy data integration platform. On the back end, an automated
  analysis pipeline processes data as it arrives off the sequencer,
  uploading the results back into Galaxy.
  \paragraph*{Conclusions:} The resulting system demonstrates the
  utility of interfacing community open source analysis platforms.
  It addresses an urgent need among small research support core
  facilities to easily and rapidly process high throughput data, thus
  enabling research scientists to quickly assess experimental
  results.
\end{abstract}

\ifthenelse{\boolean{publ}}{\begin{multicols}{2}}{}

\section*{Background}

Once available only through large genome centers, next-generation
sequencing machines have become accessible to a multitude of new
users. Among these are sequencing core facilities and individual labs
who require flexible interfaces for entering samples and tracking
sequencing runs. These users need a computational infrastructure to
efficiently process the large amounts of data produced by the
instruments. This dramatic change in scale is enabled through
utilization of automated processing pipelines, which make sequencing
results available to biologists in a format that can be readily
queried and visualized.

High throughput sequencing operations face three basic data
challenges. The first is the acquisition of accurate sample
information from the end-user. Examples include descriptive names,
preparation details, type of work required, and methods of
payment. The second challenge is tracking and managing this
information as samples move through the different steps performed in
the sequencing lab. Typically, it is useful to track when the sample
is received, quality control assays performed, and when and how the
sample was actually sequenced. The last and perhaps greatest challenge
is to rapidly return results in an intuitively useful format to the
biologist end-user.  This includes sequencing quality metrics and
processed genome alignments, available to the user in a format
compatible with visualization tools such as the UCSC Genome Browser.

The Galaxy data integration platform \cite{goecks_galaxy:_2010} helps
ameliorate the challenges of data management and analysis by providing web-based
analysis tools aimed at biological researchers. By replicating and
expanding the tools a computational user would have available at the
commandline, biologists can visualize and process the large amount of
data produced by next generation sequencing platforms. For instance,
large multiple gigabyte alignment files can be viewed directly in the
UCSC genome browser \cite{fujita_ucsc_2011} from Galaxy, allowing
biologists to immediately assess their sequencing data and view it in the context of 
other genome features. In combination with the analysis tools and
visualization options, Galaxy enables reproducible research and
collaboration between computational scientists and non-programmers by
incorporating multi-tool workflows and sharing of computational results.

A number of bioinformatics tools have emerged to help with handling
high throughput sequencing data. Sequence alignment programs such as
Bowtie \cite{langmead_ultrafast_2009} and BWA \cite{li_fast_2009}
map reads to sequenced genomes. Toolkits such
as samtools \cite{li_sequence_2009} and Picard \cite{_picard_????}
allow manipulation and querying of resulting alignment files. Larger
tool suites such as the Genome Analysis Toolkit
\cite{mckenna_genome_2010} work with aligned data to characterize
genetic variation.

A common theme of all these tools is a requirement for developing
automated workflows. For instance, a variant calling pipeline starting
from sequencing data requires more than a dozen individual
programs. Multiplexed sequencing samples further increase the need for
automation by requiring deconvolution of barcoded samples prior to
starting an analysis workflow.

The solution to these challenges is to capture experimental data,
including barcodes and run data, in a structured format. The move from
Excel spreadsheet and lab notebooks to database tracking allows development of the
necessary automation, so that data is provided to researchers in a
timely, organized manner alongside metrics of sequencing performance.
By embedding the results within the existing Galaxy toolkit, results
are immediately available for further analysis and visualization by
biologists.

Here we describe a fully integrated solution that tackles these
issues and takes full advantage of existing software and
infrastructure. The resulting tools effectively improve communication
between computational and laboratory researchers, enabling scientific work.

\section*{Implementation}

Building an automated system for processing next-generation sequencing
data requires two components. The first is a front-end web interface,
tightly integrated with Galaxy, that provides a graphical method to
quickly collect and manage sample data. The second is a back-end
pipeline which drives analysis based on sample data, reintegrating
the results with the Galaxy front-end for researcher analysis.

\subsection*{Front-end Galaxy interface}

\subsubsection*{Researcher sample entry}

Biologists use a local Galaxy server as an entry point to submit
samples for sequencing. This provides a familiar interface and central
location for both entering sample information and retrieving and
analyzing the sequencing data.

Practically, a user begins by browsing to a sample submission
page. There they are presented with a wizard interface which guides
them through entry of sample details. Multiplexed samples are
supported through a drag and drop interface (Figure 1).

When all samples are entered, the user submits them as a sequencing
project. This includes billing information and a project name to
facilitate communication between researchers and the core group
about submissions (Figure 2). Users are able to view their submissions
grouped as projects and track project status. The interface
allows support for additional services associated with sequencing,
like library construction, quantitation and validation. This is a
valuable way for users to track and organize the status of their
samples.

\subsubsection*{Sequencing tracking and management}

Sequencing lab staff have access to additional
functionality to help manage the internal sample preparation and
sequencing workflow. Samples are tracked through
a set of queues; each queue represents a state that a sample can be
in. Samples move through the queues as they are processed, with
additional information being added to the sample at each step. For
instance, a sample in the ‘Pre-sequencing quantitation’ queue moves to
the ‘Sequencing’ queue once it has been fully quantitated, with
quantitation information entered by the sequencing technician during
the transition.

Assigning samples to flow cells occurs using a drag and drop jQuery UI
interface. The design is flexible to allow for placing samples across
multiple lanes or multiplexing multiple barcoded samples into a single
lane (Figure 3).

\subsubsection*{Viewing sequencing results}

Our interface provides several ways to monitor sequencing results.
A table shows raw cluster and read counts for each sample, with
optional export functionality.  Interactive plots of run results over
time and pass rates versus read density reveal longer term performance
trends and expose individual outlier runs. These allow adjustment of
experimental procedures to maximize useful reads based on current
machine chemistry (Figure 4).

The output of processed sequencing runs -- FASTQ reads, alignments,
summary PDFs and other associated files -- are uploaded back
into into Galaxy Data Libraries organized by sample names (Figure
5). Data are only accessible by the user account that submitted the
work request, maintaining data privacy. Users can download results for
offline work, or import them directly into their Galaxy history for
further analysis and display.

\subsection*{Back-end analysis pipeline}

A pipeline written in the Python programming language manages a fully
automated analysis of sequencing results (Figure 6). Runs are detected
as they finish and reads are processed into standard FASTQ
format. Details about the finished run are passed on to storage and
analysis servers using AMQP messaging. The storage server receives
files from the sequencing machine for long term archival and backup
purposes. The analysis server receives FASTQ files in preparation for
detailed processing.

Different analysis methods can be selected by the user at the time of
submission, depending on the sequencing inputs and experimental
goals. The most complete pipeline is for variant calling, which
demonstrates the major features of the system:

\begin{itemize}
  \item De-multiplex barcoded samples sequenced in a pool, combining
    samples run on multiple lanes \cite{cock_biopython:_2009}.
  \item Short-read alignment with Bowtie or BWA
    \cite{langmead_ultrafast_2009,li_fast_2009}.
  \item Generation of alignment and read statistics with Picard and
    FastQC \cite{_picard_????,fastqc}.
  \item Preparation of a summary PDF with detailed statistics
    about the run and alignment.
  \item Recalibration and realignment using the Genome Analysis
    Toolkit
    \cite{mckenna_genome_2010,_pysam_????,gautier_intuitive_2010}.
  \item Variant identification with the Genome Analysis Toolkit's
    Unified Genotyper.
  \item Variant effect prediction using snpEff \cite{_snpeff_????}.
\end{itemize}

Lane and sample processing is fully parallelized, allowing scaling on
both single multicore machines and shared filesystem clusters. As
next-generation sequencing runs produce increasing numbers of reads in
less time, parallel analysis pipelines will be essential to scale with
throughput.

Upon completion, the finalized results are
automatically uploaded into organized Galaxy Data Libraries using the
existing Galaxy API, completing a fully automated analysis process.

The pipeline is completely general and driven by open-source
tools. Software and analysis parameters are adjustable through a
separate configuration file. Additional analysis pipelines for
experimental approaches like ChIP-seq or RNA-seq could be readily
integrated into the framework, reusing all of the existing
functionality.

\section*{Results and Discussion}

\subsection*{Collaborative development}

A unique aspect of this project is that it was undertaken
independently by the Bioinformatics core at Massachusetts General
Hospital, with the front-end code building directly on sample tracking
work done by the Galaxy team (Taylor et al. submitted concurrently).
This distributed development style
amongst non-affiliated research groups is unusual in biological software
development due to it being traditionally difficult to move into an
unfamiliar code base and provide larger scale improvements. In our
case it was helped by a transparent open-source development
methodology on the part of the Galaxy development team. For the team
at Massachusetts General Hospital, our main challenge was adopting
the flexibility to design within an existing system.

The two teams interacted primarily via e-mail and benefited from open
development tools and a structured software architecture. The
distributed revision control system Mercurial (hosted at
\url{http://bitbucket.org} was used to maintain a separate development
branch and manage merging of code changes. Additionally, the use of
the standard Model-View-Controller architecture in Galaxy allowed the
MGH team to rapidly become familiar with code and productive in
implementing changes.

We hope this type of community code development will become
increasingly common. With the continuing growth of biological data,
both through next generation sequencing and future high throughput
data generation methods, computational biologists will need to become
comfortable managing and analyzing large volumes of experimental
data. One productive way to meet this challenge is to better train
ourselves to build off existing frameworks, utilize open-source tools,
and make our code more interoperable.

\subsection*{Enabling biological research}

The largest beneficiary of this work is the biological
researcher. By establishing a structured intake
system and automated analysis pipeline, retrieving initial results
does not require manual work by bioinformaticians. As a result
the time to expect an analysis of a next-generation sequencing run has
fallen to only the computational time required for the alignments and
post-processing. The move away from a model where processing would
require manual intervention opens up entirely new classes of
experiments and accelerates the time between designing an experiment
and determining whether the data fits the expected hypotheses.

Structured input and automated processing ensures that analysis steps
are fully traceable and reproducible. Building off this core principle
of the Galaxy platform encourages reuse of scientific results, as well
as of the underlying analysis code.

\subsection*{Bioinformatics core efficiency}

An additional benefit of the automated approach is freeing
computationalists from the day-to-day management of
sequencing data. It is easy for a small group to move into
firefighting mode as data production overwhelms the ability to
analyze it. With researchers handling data entry and automated systems
taking care of initial processing tasks, computational biologists are
freed to focus on algorithm development, higher level analyses and
specific visualizations. This transformation allows computational team
members to explore more intricate analysis approaches, and overall
makes a core team able to more effectively contribute to research
output.

\section*{Conclusions}

We describe a web based information management system for
next-generation sequencing data. By developing the entry and
management forms inside of Galaxy, researchers get advanced analysis
tools along with sequencing data and analysis results. The structured
data collection of sample and sequencing information allowed the
development of an automated back end system which moves results from
the sequencer through specialized analysis pipelines. The resulting
infrastructure has fundamentally improved our ability to collaborate
with laboratory researchers.

This functionality was uniquely developed by an informal collaboration
between a bioinformatics core group and the Galaxy development
team. By combining several open-source toolkits and algorithms, the
resulting system can be readily adopted by other facilities. The
flexibility and maintainability of the code base is designed to make
extending and using this software a joy.

\section*{Availability and requirements}

The front-end entry system is available from
\url{https://bitbucket.org/chapmanb/galaxy-central}; see
\url{https://bitbucket.org/galaxy/galaxy-central/wiki/LIMS/nglims} for
instructions on configuration and activation.

The back-end analysis pipeline is available from
(\url{https://github.com/chapmanb/bcbb/tree/master/nextgen}) along
with installation instructions.

The software is targeted for UNIX/Linux and Mac OS X platforms. A
recent version of Python 2 (2.6+) is required. The software is
freely available for academic and commercial use under the
permissive MIT license.

\section*{Competing interests}

The author(s) declare that they have no competing interests.

\section*{Authors contributions}

BAC conceived and wrote the software.
MB conceived and oversaw the software development.

\section*{Acknowledgements}
  \ifthenelse{\boolean{publ}}{\small}{}
  Tammy Gillis, Shangtao Liu and Kaleena Shirley provided
  valuable insight on the web interface and laboratory management
  functionality. Roman Valls Guimera and Konrad Paszkiewicz eased
  installation and configuration by patiently providing feedback
  as early adopters of the software. Discussion with Gregory Von
  Kuster, Anton Nekrutenko, James Taylor and the rest of the Galaxy
  development team helped with Galaxy interoperability.

{\ifthenelse{\boolean{publ}}{\footnotesize}{\small}
 \bibliographystyle{bmc_article}  % Style BST file
  \bibliography{nglims_galaxy} }  % Bibliography file (usually '*.bib' )

\ifthenelse{\boolean{publ}}{\end{multicols}}{}

\section*{Figures}
\subsection*{Figure 1 - Researcher sample entry}
Entry of sequencing samples is fully integrated with Galaxy. A
wizard interface with automated completion from previous samples
allows quick collection of important details. A drag and drop
interaction makes specifying multiplexed barcode details painless.

\url{http://www.youtube.com/watch?v=HGhNMeEAFV0}

\subsection*{Figure 2 - Project submission and tracking}
Related samples are submitted as a project. The interface allows
researchers to follow the state of the samples through the
preparation and sequencing process.

\url{http://www.youtube.com/watch?v=DtQG9IzpoCU}

\subsection*{Figure 3 - Sample management by sequencing technicians}
Technicians doing the sequencing work move samples through multiple
queue states, tracking important sample details. Sequencing samples
are placed on flowcells using a drag-drop interface.

\url{http://www.youtube.com/watch?v=Sjt6y1lbzVI}

\subsection*{Figure 4 - Sequencing results}
Important statistics about sequencing runs are available to
researchers and sequencing technicians. Interactive plots enable
assessment of results over time.

\url{http://www.youtube.com/watch?v=4xrtPXE7Oe8}

\subsection*{Figure 5 - Sequencing data and analysis results}
Results are readily available to users in Galaxy data libraries. This
allows easy import into Galaxy history for analysis and visualization.

\url{http://www.youtube.com/watch?v=oizqTPLRDNg}

\subsection*{Figure 6 - Back end analysis architecture}
Orientation of sequencing, storage, analysis and Galaxy servers in the
back-end system. Relationship to front-end Galaxy interface is
included.

\url{http://chapmanb.github.com/bcbb/nglims_organization.png}

\end{bmcformat}
\end{document}
